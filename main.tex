% Tesis ITAM CLASS -- version 0.1 (13 - Abr - 2015)
% Clase para las tesis del ITAM
% 
% 13 - Abr - 2015 	Victor Martinez 	victor.martinez (at) itam.mx
% LICENSE: Creative Commons SA-BY 3.0
%
%
% Este documento presenta un ejemplo de uso de la plantilla
% El estudiante es libre de modificar este archivo a su gusto
% 
\documentclass{tesisITAM}
\usepackage[utf8]{inputenc}
\usepackage{amsmath}
\usepackage{float}
\usepackage[caption = false]{subfig}
\newcommand\fnote[1]{\captionsetup{font=small}\caption*{#1}}

\title{Modelado de Series de Tiempo para la Predicción del Precio de una Acción a partir de Técnicas de Aprendizaje de Máquina}

%\title{Time Series Modeling for Stock Market Price Prediction Using Machine Learning Techniques}

\author{\textbf{DANAE SÁNCHEZ VILLEGAS}}
\degree{\textbf{INGENIERO EN COMPUTACIÓN}}
\advisor{DR. CARLOS FERNANDO ESPONDA DARLINGTON}
\year{2016}

\begin{document}

	\pagenumbering{gobble}
	\maketitle
	\publicationrights

	%%%%%%%%%%%%%%%%%%%%%%%%%%%%%%%%%%%%%%%%%%%%%%
	% ABSTRACT
	%%%%%%%%%%%%%%%%%%%%%%%%%%%%%%%%%%%%%%%%%%%%%%

	\begin{abstract}{spanish}
		Este documento presenta tres modelos de aprendizaje de máquina para la predicción del precio de la acción de WALMEX. Los modelos son: redes neuronales artificiales, máquina de soporte vectorial y red neuronal recurrente (RNR). Se utilizaron dos criterios para comparar el desempeño de los modelos: la predicción y la dirección del movimiento del precio. El modelo de RNR tuvo el mejor desempeño en ambos casos con un error cuadrático medio de 2.14 +/- 0.42.
	\end{abstract}

	\begin{abstract}{english}
		This work presents three machine learning models to predict the stock market price of WALMEX. The models are artificial neural network, support vector machine, and recurrent neural network (RNN). Two criteria were used to compare the models: the prediction and movement's direction of the price. The RNN outperformed in both criteria with a mean square error of 2.14 +/- 0.42. 
	\end{abstract}


	\selectlanguage{english} 
	\setcounter{page}{1}
	\pagenumbering{roman}

	\tableofcontents
	\listoffigures
	\listoftables
	\newpage

	\pagenumbering{arabic}
	\setcounter{page}{1}

	%%%%%%%%%%%%%%%%%%%%%%%%%%%%%%%%%%%%%%%%%%%%%%
	% CONTENT
	%%%%%%%%%%%%%%%%%%%%%%%%%%%%%%%%%%%%%%%%%%%%%%
	 \chapter{Introduction}
\label{ch:introsm}

%\begin{chapterquote}{Oscar Wilde}
	%Where there is no love, there is no understanding.
%\end{chapterquote}


Forecasting is of great interest in different areas of scientific, industrial, and commercial activities. A time series is a collection of observations made sequentially through time \cite{chatfield2000time}. For example, the temperature in a specific region, the sales of a product in a store, and the electricity consumption in a fixed time interval. Forecasting stock prices has been regarded as one of the most challenging applications of modern time series forecasting \cite{pai2005hybrid}. It involves a lot of uncertainties and its nature is essentially dynamic, non-linear, complicated, nonparametric, and chaotic.

There are suggestions that it is not possible to predict stock prices because of its random behaviour. Nevertheless, technical analysts believe that most information about the stocks is reflected in recent prices. Therefore, if trends in the movements are observed then prices can be easily predicted \cite{patel2015predicting}. There are two types of analysis which usually investors perform before investing in a stock: fundamental analysis and technical analysis.

The fundamental analysis looks at intrinsic value of stocks, performance of the industry and economy, political climate etc.. On the other hand, technical analysis is the evaluation of stocks by means of studying statistics generated by market activity. For example, past prices and volumes. Technical analysts use stock charts to identify patterns and trends that may suggest how a stock will behave in the future\cite{patel2015predicting}. This work focuses on the second type of analysis.

Financial forecasting is of great interest for investors since it helps to decide to invest in a stock or not as well as to sell it or not in order to get profits and avoid losing money. A lot of research has been done around financial forecasting. There are two types of predictive models: traditional statistical models and machine learning models.

The first type includes the moving average, exponential smoothing, and the  autoregressive integrated moving average (ARIMA) models. These kind of models are linear in their predictions of the future values- They analyze historic data and attempt to approximate future values of a time series as a linear combination of these historic data \cite{shah2014performance}. Artificial Neural Networks (ANN) and Support Vector Regression (SVR) are two machine learning algorithms which have been most widely used for predicting stock price values  \cite{patel2015predicting}. Machine Learning uses a set of samples to generate an approximation of the underling function that generated the data \cite{shah2014performance}.

In \cite{shah2014performance} they compared the performance of Multilayer feed-forward neural network and the Elman neural network as the representative of the recurrent neural networks for stock prediction. They found that the neural network is better than the Elman recurrent network and suggested that the Elman model is only used for historical data and research purposes. On the other hand, in \cite{kim2003financial} a suport vector machine (SVM) model is applied to predicting the stock price index. Their experimental results showed that SVM provides a promising alternative to stock market prediction as they outperformed back-propagation neural networks (BPN).

Researchers have also tried hybrid models, meaning a combination of two or more models that can be machine learning and/or traditional statistical models. In \cite{pai2005hybrid} a hybrid model of autoregressive integrated moving average (ARIMA) and a support vector machine model is presented to solve the stock price forecasting problem. They found that a simple combination of the individual models does not necessarily produce the best results and demonstrated great interest in the structured selection of optimal parameters of the model.

Another hybrid model is presented in \cite{rather2015recurrent}. They proposed a model constituted of two linear models: ARIMA, and exponential smoothing model; and a non-linear model: recurrent neural network. Their results confirm about the accuracy of the prediction performance of recurrent neural network compared to linear models.

Finally, this work compares the performance of two machine learning models: artificial neural network , support vector machine, and a deep learning model: recurrent neural network (RNN) with long short term memory networks (LSTM). ANN and SVM have been widely used for stock market forecasting and RNN with LSTM is a good option for problems where data is non-linear as in this case, and where the patterns are difficult to be captured by traditional models.

\section{Objectives}
Design and implement a time series model for stock market price prediction using a recurrent neural network.

Compare the performance of the recurrent neural network model to two other predictive machine learning models, specifically, a feed forward artificial neural network model and a support vector machine model.

\section{Scope}
In this study a stock market price prediction model is presented. This approach does not seek to get a number of next predictions but just the next one. Meaning, it requires a number $n$ of real past predictions and determines the $n+1$ price in the sequence. 

Another limitation is that just machine learning models were taken in account. Traditional models such as ARIMA are not compared. Finally, stock market’s movements are affected by many macro-economical factors such as political events, general economic conditions, bank exchange rates, investors’ expectations, and so on. The data used for training this models just incorporates the prices of a fixed time interval.\\\\


The rest of this work is organized as follows. In the next chapter machine learning and deep learning are introduced. First, we explain the McCulloch and Pitts's Neuron model in order to easily understand the traditional neural networks. Then, we explain support vector machine as an optimization of neural networks.

Next, we move to deep learning and study the recurrent neural networks as well as some techniques to optimize them. This includes regularization, gradient descent variants and gradient descent optimization algorithms.  

In  chapter 3, we show how we can represent stock market prices as time series vectors and we also show the architecture for the three models. Then, we exhibit and analyze  the results of the experiments, and finally we draw the conclusions at the final chapter.


	\chapter{Theoretical Framework}
\label{ch:theosm}

% \begin{chapterquote}{Herbert Simon}
% Human beings, viewed as behaving systems, are quite simple.
% The apparent complexity of our behavior over time is largely a
% reflection of the complexity of the environment in which we find
% ourselves.
% \end{chapterquote}


\section{Machine Learning}

The process of learning has long fascinated people from many different disciplines such as psychology, biology, neuroscience, computer science, statistics, mathematics, and physics. 
Machine learning is about making computers modify or adapt their actions so that these actions get more accurate. Accuracy is measured by how well the chosen actions reflect the correct ones. As there are several ways to achieve this task there are different types of machine learning explained as follows \cite{marsland2015machine}:

\begin{itemize}
\item \textbf{Supervised Learning} a training set of examples with the correct responses or targets are provided and, based on this training set, the algorithm generalizes to respond correctly to all possible inputs.

\item \textbf{Unsupervised Learning} The algorithm tries to identify similarities between the inputs so that the inputs that have something in common are categorized together. In this type of learning correct responses are not provided.

\item \textbf{Reinforcement Learning} The algorithm gets told when the answer is wrong but does not get told how to correct it. Instead, it has to explore and try out different possibilities until it works out how to get the right answer. 

\item \textbf{Evolutionary learning} In this type of learning, biological evolution is seen as a learning process where the organisms adapt to improve their survival rates and chance of having offspring in their environment. It uses an idea of fitness which corresponds to a score for how good the solution is. 
\end{itemize}

\section{McCulloch and Pitts Neurons}
McCulloch and Pitts modeled an artificial neuron as a mathematical model in order to extract only the essentials required to accurately represent the entity, removing all the obscured details. They took in account three basic elements: a set of weighted inputs ($w_i$), an adder that sums the input signals, and an activation function that decides if the neuron fires to the current inputs.
\begin{figure}[h]
\centering
 
\includegraphics[width=10cm,height=5cm]{model_of_neuron.jpg}
\caption{McCulloch and Pitt's Neuron Model}
\label{fig:neuron}
\begin{minipage}{12cm}
    \footnotesize
    \center
    \emph \\ Taken from \cite{marsland2015machine}
    \end{minipage}
\end{figure}

The signals are added and if the sum is greater than a threshold $\theta$, it is activated. The mathematical expression is as follows:\\
\begin{equation} \label{eq:neuron}
h=\sum_{i=1}^{m} w_i * x_i + b
\end{equation}\ref{eq:neuron}

Thus the output of the neuron is the sum of the $m$ inputs ($x_i$) multiplied by the weights ($w_i$). Furthermore, units can be given biases by introducing an extra input to each unit which always has a value of 1. The weight on this extra input is called the bias (b) and is equivalent to a threshold of the opposite sign \cite{polk2002cognitive}. 


\section{Neural Networks}

A simple neural network is shown in Figure \ref{fig:nn}. It is formed by input units at the bottom, any number of intermediate or hidden layers, and a layer of output units at the top. Each of the units are  defined as a McCulloch and Pitts neuron. It should be noticed that connections within a layer from a higher to lower layer are forbidden. 

Since everything but the weights is known, learning or training refers to finding a set of weights so that for each input vector, the output vector produced by the network is sufficiently close to the desired output vector \cite{polk2002cognitive}. Thus, learning is translated to an optimization problem.

Gradient descent is an optimization algorithm used to find the parameters  that minimize a cost function. The derivative of the cost function with respect to the parameters is calculated and then equals zero. To calculate the derivative we can use backpropagation that is nothing more than the recursive application of the chain rule.
%If the function is differentiable with respect to its parameters as in this case, gradient descent is a relatively efficient optimization method. This is because the computation of first-order partial derivatives with respect to all parameters is of the same computational complexity as just evaluating the function \cite{kingma2014adam}. 

In this case, the derivative with respect to the parameters turns out to be the error defined as the difference between the actual and the desired output vectors for every case. Now the problem is that there is not a direct solution for the partial derivatives with respect to the weights equals zero. Therefore, we can not use the gradient descent for this last task but we can use a variant: stochastic gradient descent (SGD).
%escribir bien lo de igualar a cero
SGD updates the weights to the direction of the gradient. The error E in equation \ref{eq:error} is multiplied by a parameter called the learning rate $\nu$ and by the input $x_i$. This value is added to the weight ($w_ij$) for each j neuron and input i.

\begin{equation}
\label{eq:error}
E=(t_j - y_i)
\end{equation}

\begin{equation}
\label{eq:weight}
w_{ij} \leftarrow w_{ij}+\nu E * x_i
\end{equation} 
\centerline{where $0>=\nu<=1$ \\}

\begin{figure}
\center
\includegraphics[width=10cm,height=5cm]{NN.jpg}
\caption{Neural Network}
\label{fig:nn}
\end{figure}

Since the function is concave in W, the network is expected to get better answers until eventually its performance stops improving.

How much the weights are changed is controlled by the learning rate $\nu$. If it is too big, the weights will change a lot  making the network unstable so that it never settles down. On the other hand, if $\nu$ is too short it will be stable and resistant to errors and inaccuracies in the data, but it will take too long to learn. 

%The typical learning rate is between 0.1 and 0.4 \cite{marsland2015machine}.

\section{Support Vector Machines}
Support vector machine (SVM) can be seen as an optimization of the neural networks. It uses the maximum margin hyperplane which gives the maximum separation between the decision classes. Meaning. SVM uses a linear model to implement nonlinear class boundaries through some nonlinear mapping the input vectors x into a high-dimensional feature space. Then, a linear model constructed in the new space can represent a nonlinear decision boundary in the original space. Finally, in the new space, an optimal separating hyperplane is constructed \cite{kim2003financial}. 

The training examples that are closest to the maximum margin hyperplane are called support vectors and are the only ones used to define the binary class boundaries. Unlike neural networks, the only parameter to tune is the upper bound C for the non-separable cases in linear SVM \cite{drucker1999support}. Also, overfitting is unlikely to occur since it is usually caused by too much flexibility in the decision boundary and, in this case, the maximum hyperplane is relatively stable and gives little flexibility \cite{zhang1998forecasting}

The classifying rule is as follows:

\centerline{$\overleftarrow{w} * \overleftarrow{x_i} > 0  \rightarrow  positive $ \\}
\centerline{$\overleftarrow{w} * \overleftarrow{x_i} < 0  \rightarrow  negative $\\} 

The borders are calculated considering: \\
\centerline{ $\overleftarrow{y_i} (\overleftarrow{w}*\overleftarrow{x_i}+b)-1>=0$ \\}
where $y$ identifies the class $[1,-1]$\\

Therefore, the support vectors are defined as follows: \\
\centerline{$\overleftarrow{y_i} (\overleftarrow{w}*\overleftarrow{x_i}+b)-1=$\\}

To maximize the distance between the borders of the classes for linearly separable classes we need to solve the next problem:

\centerline{$min \frac{1}{2}||\overleftarrow{w}||^2$\\}
\centerline{subject to $\overleftarrow{y_i} (\overleftarrow{w}*\overleftarrow{x_i}+b)-1>=0$ \\}

In case the classes are non linear, a looseness variable $C$ should be added: \\
\centerline{$min \frac{1}{2}||\overleftarrow{w}||^2+C\sum_{i}\xi_i$\\}
\centerline{subject to $\overleftarrow{y_i} (\overleftarrow{w}*\overleftarrow{x_i}+b)-1+\xi_i>=0$\\}
\centerline{$\xi_i>=0$}

Finally, the kernel is a function that transforms the data to a higher dimension. It substitues the dot product between two vectors to build an extension of the observations. There are a few kinds of kernels. the main ones are the polynomial kernel( \ref{eq:polyKernel}) and the gaussian (\ref{eq:gaussKernel}) kernel.
\begin{equation} \label{eq:polyKernel}
K(X,X')=(X.X'+c)^degree
\end{equation}

\begin{equation} \label{eq:gaussKernel}
K(X,X')=e^(-\gamma ||X-X'||^2)
\end{equation}

%basar en apuntes de minería y buscar documentación
\section{Deep Learning}
As explained before, learning in the context of artificial neural networks refers to finding the adequate weights that make the network exhibit the desired behavior. Depending on the problem and how the neurons are connected, such behavior may require long causal chains of computational stages. Each stage transforms the aggregate activation of the network.

Deep Learning is about accurately assigning credit across many such stages \cite{schmidhuber2015deep}. This work will focus on recurrent neural networks, a subfield of deep learning in artificial neural networks (NNs).
%Another way of understanding deep learning is by studying neural networks as a representation learning method. This kind of methods allow a machine to be fed with raw data and to automatically discover the representations needed for detection or classification. Hence, deep learning methods are representation learning methods with multiple levels of representation. With the composition of enough transformations, very complex functions can be learned \cite{lecun2015deep}. 

\subsection{Recurrent Neural Networks}

A deep recurrent neural network (RNN) is a deep artificial neural network but with a different structure. In the deep case, connections among hidden units are allowed. Furthermore, these connections are associated with a time delay enabling the model to retain information about the past inputs. In this way, temporal correlations between events that are possibly far away from each other in the data can be discovered \cite{pascanu2013difficulty}.

\begin{figure}
\center
\includegraphics[width=3.5cm,height=5cm]{rnn.png}
\caption{Recurrent Neural Network}
\label{fig:r_nn}
\begin{minipage}{12cm}
    \footnotesize
    \center
    \emph \\ Taken from \cite{lecun2015deep}
    \end{minipage}
\end{figure}
For tasks that involve sequential inputs it results good to use RNNs. This is mainly because RNNs process an input sequence one element at a time and maintain in their hidden units a "state vector". This vector implicitly contains information about the history of all the past elements of the sequence.\cite{lecun2015deep} 

Sequential data arises through measurement of time series such as the rainfall measurements on successive days at a particular location,the daily values of a currency exchange rate and, to our interest, the stock market price in a time interval.The order of the prices as well as the relationships between them matter. Thus, to predict which is the most likely price to follow the sequence, it is important to retain this information.

% In this case the input data is the set of elements of a sentence or a set of sentences and the desired behaviour is to generate a new sequence that can adequately continue after the original one. Therefore, the target data for each element is the following one. 
The recurrent neural network receives an input vector $x=[x_1,...,x_T]$ which is passed through weighted connections to N recurrently connected hidden vector sequences $h^n=[h_1^n,...,h_T^n]$ and finally to the output vector sequence $y=[y_1,...,y_T]$ which will be compared to the target value for each observation. To compute the hidden layer activation the following equations are iterated from t=1 to T and from n=2 to N. \cite{graves2013generating}:

\begin{equation} \label{eq:hidden1}
h_t^1= H( W_{ih^1} * x_t + W_{h^1 h^1}*h^1_{t-1} + b^1_h)
\end{equation}

\begin{equation} \label{eq:hidden}
h_t^n= H(W_{ih^n} * x_t + W_{h^{n-1}  h^n} * h^{n-1}_t +W_{h^n h^n} * h^n_{t-1}+ b^n_h)
\end{equation}

The W terms denote the weight matrices that connect the input layer to the first hidden layer, two hidden layers between them or a hidden layer to an output layer. $H$ is the hidden layer function which usually is an application of a sigmoid function like the hyperbolic tangent. Finally, given the hidden sequences, the output sequence is computed as follows:

\begin{equation} \label{eq:output}
y_t=Y(\sum_{n=1}^{N} W_{h^{n}y} * h^n_t + b_y)
\end{equation}

Where $Y$ is the output layer function whose output vectors will need to be compared to the target values so that the network achieves the desired behaviour by minimizing the difference between both values.
%In Equation \ref{eq:error} the error was defined as this difference. Nevertheless, this definition is not adequate
The network now can be trained using the gradient descent to learn how the parameters, meaning the $W$ vector, should change to decrease the loss  $L(x)$. Therefore, the partial derivatives of the loss with respect to the weights should be computed. This can be achieved using backpropagation through time (BTT). This process computes the gradients of expressions through recursive application of the chain rule.

\subsubsection{Backpropagation Through Time}
BTT applied to RNNs have the constraint of sharing the parameters within each layer. As showed in  \ref{fig:unfold}, RNNs once unfolded in time can be seen as very deep feedforward networks in which all the layers share the same weights\cite{lecun2015deep}. Here, the model is presented as a deep multi-layer where each time step in the interval $[t,T]$ is a layer with N units each \cite{pascanu2013difficulty}. 

The constraint can be satisfied by adding the gradients for W at each time step. This constraint can also be seen as an advantage since the network requires less number of parameters than a non-recurrent neural network.

\begin{figure}

\center
\includegraphics[width=10cm,height=5cm]{unfold.png}
\caption{Unfolded Recurrent Neural Network}
\label{fig:unfold}
\begin{minipage}{12cm}
    \footnotesize
    \center
    \emph \\ Taken from \cite{lecun2015deep}
    \end{minipage}
\end{figure}

%The key insight in BTT is that the calculation of the derivative of the objective with respect to the input of a module can be done by working backwards from the gradient with respect to the output of that module and repeating this process through all the layers or, in this case, through all the time steps \cite{schmidhuber2015deep}.

%a lo mejor aquí poner lo del -1

\subsubsection{RNNs Training Issues}
As described before, RNNs are very powerful dynamic systems. Unfortunately training them has proved to be problematic. During the gradient backpropagation phase, the gradient signal can end up being multiplied a large number of times by the weight matrix associated with the connections between the neurons of the recurrent hidden layer. This causes that the backpropagated gradients either grow or shrink at each time step. Thus, the magnitude of weights in the transition matrix may have a strong impact on the learning process.

If the  weight matrix is smaller than 1, the gradient signal can get so small that so over many time steps it can vanish. Conversely, if the eigenvalue of the weight matrix is larger than 1, it can lead to a situation where the gradient signal is so large that it can explode causing learning to diverge. 

% The exploding gradients problem refers to the large increase in the norm of the gradient during training caused by the explosion of the long term components. On the other hand, the vanishing gradients problem is presented when long term components go exponentially fast to norm 0 making impossible for the model to learn correlation between temporally distant events \cite{pascanu2013difficulty} which was the main reason for choosing RNNs for this work. 

\subsubsection*{The exploding gradients problem}
 The fact that the norm of the gradient increases during training is a problem because the computation becomes inefficient and also increases the response time. One simple mechanism to deal with this problem is to rescale it whenever it goes over a threshold. To choose wisely the threshold, one good heuristic is to look at statistics on the average norm over a sufficiently large number of updates. In this way, we can handle very abrupt changes in norm as it adapts the learning rate based on the norm of the gradient  \cite{pascanu2013difficulty}.

\subsubsection*{The vanishing gradients problem}

The vanishing gradients problem makes it difficult to learn to store information. Having a longer memory has a establishing effect, because even if the network cannot make sense of its recent history, it can look further back in the past to formulate its predictions. \cite{graves2013generating}

%This amnesia makes them prone to instability when generating sequences. The main problem is that if the network's predictions  are only based on the last few inputs, and these inputs were themselves predicted by the network, it has little opportunity to recover from past mistakes.

A lot of research work has been made to propose solutions to this problem. The main idea is to augment the network with an explicit memory. The first proposal of this kind is the long short-term memory (LSTM) network used in this work. The LSTM has complicated dynamics that allow it to easily “memorize” information for an extended number of time steps \cite{zaremba2014recurrent}.

\subsubsection{Long Short-Term Memory Networks}
Long Short-term Memory (LSTM) is an RNN architecture. The difference is that instead of using a sigmoid function as hidden layer function, LSTMs use memory cells to store information. The new architecture has three gates and a memory cell. The gates are for input, forget, and output functions. The memory cell has a connection to itself at the next time step with a weight of one, meaning it copies its own real-valued state and accumulates the external signal \cite{lecun2015deep}. Thus, it can decide to overwrite the memory cell, retrieve it, or keep it for the next time step\cite{zaremba2014recurrent}. 

LSTMs are better at storing and accessing information than standard RNNs. The cells improve the performance when finding and exploiting long range dependencies in the data. Additionally, the full gradient can be calculated with backpropagation through time as well.\cite{graves2013generating}

\begin{figure}
\center
\includegraphics[width=10cm,height=5cm]{gates.PNG}
\caption{Long Short Term Memory Network}
\label{fig:lstm}
\begin{minipage}{12cm}
    \footnotesize
    \center
    \emph \\ Based on \cite{greff2016lstm}
    \end{minipage}
\end{figure}

The forget gate $f_t$ decides the content to keep for each time step $t$. It is a sigmoid function that takes in account the $h_{t-1}$ and the present input $x_t$ and returns a number between 0 and 1 where 0 means "forget it" and 1 means "keep it". 

The next step is to calculate the input gate to decide which values to update using a sigmoid function (Eq. \ref{eq:input}). Then, the new candidates are computed by a tanh function (Eq. \ref{eq:candidate}), and the memory cell update is done by combining the previous functions as explained in Eq. \ref{eq:update}.

\begin{equation} \label{eq:forget}
f_t=\sigma(W_f*(h_{t-1},x_t)+b_f)
\end{equation}
\begin{equation} \label{eq:input}
i_t=\sigma(W_i*(h_{t-1},x_t)+b_i)
\end{equation}
\begin{equation} \label{eq:candidate}
C'_t=tanh(W_C*(h_{t-1},x_t)+b_C)
\end{equation}
\begin{equation} \label{eq:update}
C_t=f_t*C_{t-1}+i_t*C'_t
\end{equation}

Finally, the output function decides which parts of the memory cell to return with another sigmoid function (Eq. \ref{eq:output}), and lastly the $h_t$ is calculated (Eq. \ref{eq:hidden}) with the output function and a tanh function of the memory cell to push the values into the interval [-1,1].

\begin{equation} \label{eq:output}
o_t=\sigma(W_o*(h_{t-1},x_t)+b_o)
\end{equation}
\begin{equation} \label{eq:hidden}
h_t=o_t*tanh(C_t)
\end{equation}

\subsection{Regularization}

Successful applications of neural networks require good regularization. Although deep neural networks with a huge number of parameters are very powerful machine learning systems, they tend to overfit at training time causing an error with great variance. 

Fortunately, dropout is a technique for addressing this problem. The key idea is to randomly drop units (along with their connections) from a neural network during training in order to prevent the units from co-adapting too much \cite{srivastava2013improving}. However, when working with recurrent neural networks and specifically with LSTM networks it is not desired to erase all the information from the units. It is extremely important that the units remember events that occurred many time steps in the past. This is the reason for \cite{zaremba2014recurrent} to only use drop out in the non-recurrent connections. 

Nevertheless, in \cite{gal2015theoretically} they develop a technique to effectively use drop out in all connections called Variational RNN. In this dropout variant, the same dropout mask at each time step is repeated for both inputs, outputs, and recurrent layers, so the same network units are dropped at each time step. 
%The interpretation for this work (character-level language modelling) is to force the model not to rely on single characters for its task.
%Finally, they found that the optimal dropout probabilities are between 0.3 and 0.5.

\subsection{Gradient Descent Variants}
As previously introduced, gradient descent is a an optimization algorithm used to find the parameters that minimize an objective function. There are three variants of the gradient descent: batch gradient descent, stochastic gradient descent, and mini-batch gradient descent. The variants differ in how much data is used to compute the gradient of the objective function. 

The first variant computes the gradient of the cost function with respect to the parameters($\theta$) for the entire dataset as in \ref{eq:batchgd}.

\begin{equation} \label{eq:batchgd}
\theta=\theta-\eta * \Delta_\theta * J(\theta)
\end{equation}

The second variant computes a parameter update for each training example $x_i$ and target $y_i$ \ref{eq:sgd}.
\begin{equation} \label{eq:sgd}
\theta=\theta-\eta * \Delta_\theta; * J(\theta;x_i;y_i)
\end{equation}

Where $\eta$ is the learning rate.

The problem with the batch gradient descent is that it can be very slow and it does not allow to update the model with new examples on-the-fly. The stochastic gradient descent solves this problem and is faster than the batch gradient descent, but it performs updates with high variance causing the objective function to fluctuate heavily.  

Finally, the mini-batch gradient descent performs an update for every mini-batch of $n$ training examples as in \ref{eq:mbgd} \begin{equation} \label{eq:mbgd}
\theta=\theta-\eta * \Delta_\theta * J(\theta;x_{i:i+n};y_{i:i+n})
\end{equation}

In this way, the mini-batch gradient descent reduces the variance of the updates, is faster than the first variant and allows the model to update with examples on-the- fly. Therefore, this third option is the chosen one for this work\cite{ruder2016overview}. 

\subsection{Gradient Descent Optimization Algorithms}
%To our knowledge, mini-batch Gradient Descent (SGD) is the best option to find the parameters that minimize an objective function. Mini-batch gradient descent is stochastic since the objective function is composed of a sum of subfunctions evaluated at different subsamples of data. Another source of noise may also come from dropout regularization \cite{kingma2014adam}. 
Choosing the proper learning rate $\eta$ is a difficult and important task. Using the same learning rate for all parameter updates may not be the best choice if the data is sparse and the features have different frequencies. In this case, it would be better to perform a larger update for rarely occuring features. Another challenge is to avoid getting trapped in a local minima when working with non-convex objective functions as is the usual case in deep learning. For this cases, efficient stochastic techniques as the ones explained in the following are required \cite{ruder2016overview}.
%fuente:http://sebastianruder.com/optimizing-gradient-descent/

The main algorithms used to avoid those challenges in deep learning applications are: Momentum, Nesterov and the accelerated gradient, Adagrad, RMSPop, and Adam. The first two algorithms adapt the updates to the slope of the error function and, thus speed up SGD. The last ones, adapt the updates to each individual parameter to perform larger updates for infrequent and smaller updates for frequent parameters. This makes the algorithm well-suited for dealing with sparse data \cite{ruder2016overview}.

%Momentum \cite{qian1999momentum} is a method that helps accelerate gradient descent in the relevant direction and dampens oscillations by taking in account a fraction $\phi$ of the update vector of the past time step to the current update vector:

%\begin{equation}
%\theta=\theta-(\nu_t=\phi \nu_{t-1} + \eta \Delta_\theta J(\theta))
%\end{equation}
%Where $\phi$ is the momentum term usually set to 0.9.

%The momentum term increases for dimensions whose gradients point in the same directions and reduces updates for dimensions whose gradients change directions. As a result, we gain faster convergence and reduced oscillation.
%chance y quitar esto

%Nesterov accelerated gradient \cite{nesterov1983method} not only uses the momentum term $\phi \nu_{t-1}$  to move the parameters $\theta$; but it also computes $\theta - \phi \nu_{t-1}$ to give an approximation of the next position of the parameters.  
%\begin{equation}
%\theta=\theta-(\nu_t = \phi \nu_{t-1} + \eta \Delta_\theta J(\theta - \phi \nu_{t-1})
%\end{equation}
  
The name Adam comes from adaptive moment estimation. It combines the advantages of AdaGrad and RMSProp \cite{kingma2014adam}. As RMSprop, ADAM  store an exponentially decaying average of past squared gradients $\nu_t$: %In addition, Adam keeps an exponentially decaying average of past gradients $m_t$ similar to momentum:
\begin{equation}
m_t=\beta_1 m_{t-1} + (1-\beta_1)g_t
\end{equation}

\begin{equation}
\nu_t=\beta_2 \nu_{t-1} + (1-\beta_2)g^2_t
\end{equation}

Where $m$ and $\nu$ are estimates of the first moment and the second moment of the gradients respectively, meaning the mean and the uncentered variance. Both are initialized as vectors of zeros and, thus, they are biased towards zero. Then, they are corrected as follows:
\begin{equation}
\hat{m_t}=\frac{m_t}{1-\beta^t_1}
\end{equation}

\begin{equation}
\hat{\nu_t}=\frac{\nu_t}{1-\beta^t_2}
\end{equation}

Finally, the next equation is computed to update the parameters:
\begin{equation}
\theta_{t+1}=\theta_t-\frac{\eta}{\sqrt[2]{\hat{\nu_t}} + \epsilon} \hat{m_t} 
\end{equation}

Where $\epsilon$ is a smoothing term to avoid the zero division. The default proposed values are $10^-8$ for $\epsilon$, 0.9 for $\beta_1$ and 0.999 for $\beta_2$.
%falta fuente

%Adam is better than Adagrad because it resolves the main disadvantage of it that is its accumulation of the squared gradients in the denominator that divides the learning rate\cite{duchi2011adaptive}. Since every added term is positive, the accumulated sum keeps growing during training causing the learning rate to shrink and eventually becomes infinitesimally small, so the algorithm can not acquire additional knowledge\cite{ruder2016overview}. 

It has been empirically shown that Adam works well in practice and better than other adaptive-learning algorithms including the RMSProp and AdaGrad, and therefore it is the one chosen for this work.
%fuente!!!!!!!!!!!!!!!!!!!!!!!!!!!!!!!!!!!!!!!!!!!!!!!!!
-

    \chapter{Development}
\label{ch:dev}

%\begin{chapterquote}{Ludwig Wittgenstein}
%	The limits of my language mean the limits of my world.
%\end{chapterquote}
In this section we will explain the data used for the experiments as well as the process of preprocessing.

\section{Data description}
The data used in this study includes the closure stock market prices of \textit{Wal-Mart de México, S.A.B. de C.V.} known as WALMEX. The data was extracted from Yahoo! Finance applying a weekly filter over a time interval: from 02/01/2012 to 16/01/2017. The result is a database with 264 weekly prices in MXN currency. The average price is \$32.72 MXN with a standard deviation of 4.32. The minimum price is \$28.09 MXN and the maximum is \$47.22 MXN. Figure \ref{fig:walmexFreq} shows the frequency of the prices by \$1 MXN interval since the data is too variable to be able of analyzing the frequency of each price. It can be seen that the interval of highest frequency includes the prices between \$34.5 to \$35.4. 

On the other hand, if we analyze the prices by the average price of each month and year it can be noticed that the years can be classified in four different intervals. Nevertheless there are no two following years that keep the same average price interval.In 2014 the lowest average price was registered and the highest appears in 2016. When analyzing by month it can be seen in Figure \ref{fig:walmexMonth} that the months can be grouped in three intervals. In this case, there are following months that remain in the same interval: from May to July and August to September. 

 
\begin{figure}
\label{fig:walmexComp}
\center
\includegraphics[width=8.5cm,height=3.5cm]{Figures/wamexCompleto.PNG}
\caption{Weekly WALMEX Stock Prices}
\end{figure}

\begin{figure}
\label{fig:walmexFreq}
\center
\includegraphics[width=8.5cm,height=3.5cm]{Figures/walmexHistogram.PNG}
\caption{Frequency of Prices by Intervals}
\end{figure}

\begin{figure}
\label{fig:walmexMonth}
\center
\includegraphics[width=8.5cm,height=3.5cm]{Figures/walmexMonth.PNG}
\caption{Monthly Average Stock Price}
\end{figure}

\begin{figure}
\label{fig:walmexYear}
\center
\includegraphics[width=8.5cm,height=3.5cm]{Figures/walmexYear.PNG}
\caption{Yearly Average Stock Price}
\end{figure}
\section{Preprocessing}

In this section we explain how prices are mapped to an adequate representation that the three models understand. For this task, data needs to be vectorized and we need to define the target values since the three models proposed belong to supervised learning. In this study, the preprocessing took in account three steps:

\begin{enumerate}
\item Data split
\item Normalization
\item Vectorization
\end{enumerate}

First, the database was divided in two subsets: 67\% of the data for training and the remaining 33\% for testing. Since we are working with time series, the order of the values was respected. Therefore, the first 177 prices were assigned to the training subset and the last 88 values to the testing subset. From this point, the testing subset was not occupied until the evaluation phase.

The next step was to normalize the data, meaning, the data was rescaled to the interval $[0,1]$ as follows:

\begin{equation}
\label{eq:normalize}
x_{new}=\frac{x-x_{min}}{x_{max}-x_{min}}
\end{equation}

where $x_{min}$ is the minimum value in the training subset and $x_{max}$ is the maximum values in the training subset. 

For the evaluation phase, the testing subset was also normalized taking in account the maximum and minimum value of the training subset in order to avoid adding future information to the testing subset.

In this work sequences of three prices were defined as the attributes of the database, hence, the target value is the fourth one in the sequence. Since the data is already in weekly time intervals, there is no need to keep the dates for each price. This representation gave us a training set of 173 examples and a testing set of 84. The first five examples of prive vectors with its target price are shown in table \ref{table:trainPrices}. Table \ref{table:train} exhibits the vectors after normalization.


\begin{table}{}
\begin{center}
\begin{tabular}{ c | c | c | c }
    \hline
     \textbf{Price 1} &  \textbf{Price 2} &    \textbf{Price 3} &   \textbf{TargetPrice}\\ \hline
    36.88&  37.49&  38.79 &39.09\\ \hline
    37.49&  38.79&  39.09&39.79\\ \hline
    38.79&  39.09&  39.79&39.93\\ \hline
   39.09&  39.79&  39.93&39.79\\ \hline
    39.79&  39.93&  39.790&40.72\\ \hline
    \hline
  \end{tabular}
\caption{First 5 Examples of Price Vectors}
 \label{table:trainPrices}
\end{center}

 \end{table}


\begin{table}{}
\begin{center}
\begin{tabular}{ c | c | c | c }
    \hline
    \textbf{Price 1} &  \textbf{Price 2} &    \textbf{Price 3} &   \textbf{Target Price} \\ \hline
    0.55696201 &  0.59493685&  0.6778481 &0.69620258\\ \hline
    0.59493685 &  0.6778481 &  0.69620258 &0.74050641\\ \hline
     0.6778481 &  0.69620258 &  0.74050641&0.74936712\\ \hline
     0.69620258 & 0.74050641 &  0.74936712 &0.74050641\\ \hline
     0.74050641 &  0.74936712 &  0.74050641 &0.79936719\\ \hline
    \hline
  \end{tabular}
\caption{Price Vectors after Normalization}
\end{center}
\label{table:train}
 \end{table}
 
    %\chapter{Methodology}

\label{ch:methodology}

%\begin{chapterquote}{Ludwig Wittgenstein}
%	The limits of my language mean the limits of my world.
%\end{chapterquote}
This section presents the experiments for each model. First, we present the parameter selection experiments. Then, we explain the experiments for stock market price prediction as well as the metrics and criteria used to compare the models.

The three models were implemented in Anaconda with Python 2.7.12 in a 64 bits latop with 4GB of RAM. The processor of the system is Intel(R) Core(TM) i5-3337U CPU @ 1.80 GHz and the operative system is Windows 10. In order to replicate the experimental results we fixed the seed to 14. 
\section{Parameter Selection}

Each model required a parameter to be chosen. For the ANN we looked for the number of hidden layers, for the SVM the  parameter to be chosen is the C; and for the RNN we looked for the number of LSTM cells required.

Since it is a time series problem, the order of the examples must be respected. To make sure this happens, the parameters were selected using time series cross validation over the training subset. In this k-fold cross validation variant successive training sets are supersets of those that come before them. In all cases, we looked for the parameter that minimized the mean square error (MSE). 

For each cross validation the next steps were implemented:
\begin{enumerate}
\item Split the training data in $k=5$ arranged subsets
\item For each candidate parameter p:
\item  Build the model with p
\begin{enumerate}
\item For each split of $k-1=4$ length subsets:
\begin{enumerate}
\item Train the model
\item Evaluate the model using the left subset
\item Calculate and keep the MSE
\end{enumerate}
\item Calculate and keep the average and the standard deviation of the MSEs
\end{enumerate}
\item Choose the parameter with the minimum MSE
\end{enumerate}

\subsection{Artificial Neural Network}

The artificial neural model was implemented using Pybrain, a flexible machine learning library for Python. The input layer has three neurons for each real price and the output layer has only one neuron for the prediction of the next price. The minimum MSE after applying cross validation was gotten using three neurons in the hidden layer as shown in  Figure \ref{fig:cvNN}

\begin{figure}[h]
\centering
\includegraphics[width=8cm,height=4.5cm]{cvNN.PNG}
\caption{MSE for each ANN Configuration}
\caption*{The blue line shows the average of the MSEs of each network configuration. The green and the red lines represent the error lines showing the +/- standard deviation errors of the scores}
\label{fig:cvNN}
\end{figure}

\subsection{Support Vector Machine}
The support vector machine was implemented using the scikit-learn library or Python. It provides simple and efficient tools for data mining and data analysis. We used a sigmoid kernel and C=3 which got the minimum MSE as shown in Figure \ref{fig:cvSVM}

\begin{figure}[h]
\centering
\includegraphics[width=8cm,height=4.5cm]{cvSVM.PNG}
\caption{MSE for each SVM Configuration}
\caption*{The blue line shows the average of the MSEs of each network configuration. The green and the red lines represent the error lines showing the +/- standard deviation errors of the scores}
\label{fig:cvSVM}
\end{figure}

\subsection{Recurrent Neural Network}

The model was implemented in Keras, a high-level neural networks library written in Python. It can run on top of either TensorFlow or Theano and with CPU or GPU. It supports recurrent neural networks, and arbitrary connectivity schemes including multi-input and multi-output training. For this study we used Theano as background.

The model has an input, a hidden and an output layer. It has 3 outputs for each real price. The single hidden layer has 10 LSTM units with a drop out of 0.03 for the non-recurrent connections. Finally, we have a full connected layer with one output which refers to the prediction of the price. The network implements the  ADAM SGD optimizer in order to adapt the learning rate dynamically since the data is sparse. 

As shown in Figure \ref{fig:cvRNN}, the minimum MSE is nor gotten with 10 LSTMs. Nevertheless, with one more LSTM the MSE is higher and then it drops again. Since the difference between the MSE with 10 LSTMs and with 16 LSTMs is just 0.003, we decided to use 10 LSTMs. In addition, using less LSTMs requires less parameters to be calculated by the network.  

\begin{figure}[h]
\centering
\includegraphics[width=8cm,height=4.5cm]{cvRNN.PNG}
\caption{MSE for each RNN Configuration}
\caption*{The blue line shows the average of the MSEs of each network configuration. The green and the red lines represent the error lines showing the +/- standard deviation errors of the scores}
\label{fig:cvRNN}
\end{figure}

\section{Comparison Criteria}

Since we are predicting stock market prices this study is strongly interested in two main criteria:

\begin{itemize}
\item How precise is the model to predict the price
\item How precise is the model to capture the up and down movements
\end{itemize}

For the first criteria we used the MSE to compare each model performance. The second criteria required a new categoric vector for the real test values and for the predicted values of each model in order to express the ups and downs of the prices. This vector mapped a 0 if the new price was lower than the past one and 1 if it was higher. In this way we could compare the models using the accuracy, precision, recall, and F1 scores. Another resource used is the confusion matrix. 

The metrics as well as the confusion matrix are explained as follows:

\begin{itemize}
\item \textbf{MSE:} the average of the quadratic differences between the real and the predicted value
\item \textbf{Accuracy:} The proportion where the predicted values are the same as the real values out of the total of examples
\begin{equation}
MSE=\frac{1}{n}\sum_i (Y_{real}-Y_{predicted})^2
\label{eq:msemet}
\end{equation}
\item \textbf{Precision:} the number of true positives divided by the sum of True positives and false positives, meaning, the total number of elements labeled as belonging to the positive class 
\begin{equation}
Precision=\frac{TP}{TP+TN}
\label{eq:precision}
\end{equation}
\item \textbf{Recall}: the number of true positives divided by the true positives plus the false negatives, meaning, the total number of elements that actually belong to the positive class
\begin{equation}
Recall=\frac{TP}{TP+FN}
\label{eq:recall}
\end{equation}
\item \textbf{F1 Score:} it is a weighted average of the precision and recall. The F1 score reaches its best value at 1 and worst at 0 and is equals to two times the division between the product of precision and recall, and the sum of both metrics.
\begin{equation}
F1 Score=2*\frac{Precision*Recall}{Precision+Recall}
\label{eq:fiscore}
\end{equation}
\item \textbf{Confusion Matrix:} it shows the true positive, true negative, false positive and false negative values. Each column represents the examples in a predicted class while each row represents the examples in an actual class.
$$
\begin{matrix}
TP&FP\\
FN&TN\\
\end{matrix}
$$
\end{itemize}


























    \chapter{Experimental Results}
\label{ch:expResults}
\begin{chapterquote}{Felix Klein}
"The greatest mathematicians, as Archimedes, Newton, and Gauss, always united theory and applications in equal measure."
\end{chapterquote}

\section{Accuracy and MSE}
In this chapter the results for each model and each experiment are shown. We compared the models using the two criteria explained in the previous chapter. The RNN was trained for just 10 epochs to avoid overfitting as it quickly reached a very low in sample error smaller than 0.01. On the other hand, the ANN was trained for 300 epochs in order to reach approximately the same in sample error as the RNN. 


The MSE and the accuracy scores are shown in Table \ref{table:accmse}. The best performance is where the MSE is the lowest and the accuracy is the highest. It can be noticed in Figure \ref{fig:accmse} that the RNN had the best performance in both cases. The support vector machine got a higher accuracy than the ANN but the last one had a lower MSE.

In this case, the accuracy is measuring the second comparison criteria, meaning, how good is the model to catch the ups and downs of the stock market price. The best models are the RNN and the SVM. On the other hand, the RNN and the ANN are better in predicting the price. Nevertheless, there is a difference of more than 1 point between the MSE of the two models.  

%tabla de accmse
\begin{table}{}
\begin{center}
\begin{tabular}{ c | c | c }
    \hline
     \textbf{Model} &  \textbf{MSE} &    \textbf{Accuracy}\\ \hline
    RNN&  1.66&  0.45 \\ \hline
    SVM&  3.39&  0.44\\ \hline
    ANN&  2.91&  0.41\\ \hline
      \hline
  \end{tabular}
  \caption{MSE and Accuracy of the Models}
 \label{table:accmse}
\end{center}
 \end{table}

%gráfica de accmse
\begin{figure}
\label{fig:accmse}
\center
\includegraphics[width=13.5cm,height=8.5cm]{Figures/accmseBar.JPG}
\caption{MSE and Accuracy of each Model}
\end{figure}


\section{Confusion Matrix, Precision, F1 Score}

The confusion matrix, precision, and F1 score correspond to the second comparison criteria and are studied by class: Up, if the actual price is higher than the past one and Down, if it is lower. This analysis is interesting since we can study where each model is better than the other. 

The values in the left to right diagonal of the confusion matrix, are the ones that the model predicted correctly. Conversely, the values out of this diagonal are the ones where the model got the prediction wrong.

As shown in \ref{fig:resultsRNN}, the most of the RNN mistakes are at the upper right, meaning, the true negative values that correspond to the ups of the price. It can be said that the model moved to the right direction in almost half of the times. Although it may seem as a bad performance, the mean of the differences between the actual price and the previous one in the test subset is only 0.5 points. 

On the other hand, the majority of the mistakes of the SVM and the ANN models were at the false positive corner. This case is when the price went down with respect to the previous one and the model moved up. The ANN performed better in this case, but the SVM outperformed the ANN when the price went up.

%resultados RNN
\begin{figure}
\center
\subfloat[RNN Confusion Matrix]{\includegraphics[width=6.5cm,height=5.5cm]{Figures/cmRNN.JPG}} 
\subfloat[RNN Normalized Confusion Matrix]{\includegraphics[width=6.5cm,height=5.5cm]{Figures/cmnRNN.JPG}}\\
\subfloat[RNN Classification Report]{\includegraphics[width=10.5cm,height=5.5cm]{Figures/reportRNN.JPG}}
\caption{RNN Results}
\label{fig:resultsRNN}
\end{figure}

%resultados SVM
\begin{figure}
\center
\subfloat[SVM Confusion Matrix]{\includegraphics[width=6.5cm,height=5.5cm]{Figures/cmSVM.JPG}} 
\subfloat[SVM Normalized Confusion Matrix]{\includegraphics[width=6.5cm,height=5.5cm]{Figures/cmnSVM.JPG}}\\
\subfloat[SVM Classification Report]{\includegraphics[width=10.5cm,height=5.5cm]{Figures/reportSVM.JPG}}
\caption{SVM Results}
\label{fig:resultsSVM}
\end{figure}

%resultados NN
\begin{figure}
\center
\subfloat[NN Confusion Matrix]{\includegraphics[width=6.5cm,height=5.5cm]{Figures/cmNN.JPG}} 
\subfloat[NN Normalized Confusion Matrix]{\includegraphics[width=6.5cm,height=5.5cm]{Figures/cmnNN.JPG}}\\
\subfloat[NN Classification Report]{\includegraphics[width=10.5cm,height=5.5cm]{Figures/reportNN.JPG}}
\caption{NN Results}
\label{fig:resultsNN}
\end{figure}

%comparison
\begin{figure}
\center
\subfloat[Precision]{\includegraphics[width=8cm,height=5.5cm]{Figures/precision.JPG}} 
\subfloat[Recall]{\includegraphics[width=8cm,height=5.5cm]{Figures/recall.JPG}} \\
\subfloat[F1 Score]{\includegraphics[width=8cm,height=5.5cm]{Figures/F1_Score.JPG}}
\caption{Comparison between ANN, SVM, and RNN}
\label{fig:comparison}
\end{figure}

%Precision
The correct predictions were very balanced in both cases between ups and downs taking in account that the test subset had 6 more downs than ups. The recall graph from figure \ref{fig:comparison}, shows the SVM outperformed the two other models when predicting the downs of the price. Nevertheless, the RNN model was better to predict when the price was gong to be higher than the last one.

Lastly, comparing the three models by the F1 score, we get the best performance in the RNN model in both cases: ups and downs. The SVM performed similarly when predicting the downs of the price. Finally, the ANN and the SVM got almost the same F1 score when predicting the ups, but it was lower than the F1 score from the RNN. 

\section{ANN vs. RNN}
As the recurrent neural network is a modification of the artificial neural network we found interesting to analyze both models between each other more deeply. We ran each model 30 times. Then, we calculated the average of the mean square error as well as the standard deviation to evaluate each model. The results are shown in table 
\ref{table:nnRnn}.

The results show that the RNN outperformed the traditional neural network justifying the need of a more expensive and powerful structure like the RNN for this problem. The RNN achieved a lower mean square error and its bias was smaller than the ANN's.

\begin{table}{H}
\begin{center}
\begin{tabular}{ c | c | c }
    \hline
     \textbf{Model} &  \textbf{Average MSE} &    \textbf{Standard Deviation}\\ \hline
    RNN&  2.14&  0.42 \\ \hline
    ANN&  2.75&  1.22\\ \hline
      \hline
  \end{tabular}
\caption{Evaluation of Models}
 \label{table:nnRnn}
\end{center}
 \end{table}
 %gráfica de errores
\begin{figure}
\label{fig:errors}
\center
\includegraphics[width=13cm,height=7cm]{Figures/errors.JPG}
\caption{ANN MSE vs. RNN MSE}
\end{figure}

 
 

 
 

    \chapter{Conclusions}
\label{ch:conclusions}
\begin{chapterquote}{ Sam Tanenhaus}
“In literature and in life we ultimately pursue, not conclusions, but beginnings.”
\end{chapterquote}


This study presented a novel work of stock market price prediction. It involved modeling time series with machine learning techniques. We chose machine learning techniques over traditional statistics because the first ones capture the random movements and the variability better than the second ones. 


We modeled the closure stock market price of WALMEX, the largest Walmart's division outside the U.S. The three models: ANN, SVM, and RNN took into account three real prices and returned the prediction of the next price.

First, we described the data, then we normalized it, built the dataset, and finally, we split it into a train and test subsets. Since we needed to respect the order of the values in the dataset, we used time series cross-validation to find the parameters of each model.

The minimum MSE was found with 3 hidden neurons for the ANN, a C equals 3 for the SVM, and 10 LSTMs for the RNN. The ANN was trained with 300 epochs, while the RNN just required 10 since it overfits very quickly. The SVM's kernel was sigmoid. It was chosen as from the data description analysis. The data histogram in Figure \ref{fig:walmexFreq} shows an almost normal distribution of the prices. 

Two criteria were used to compare the performance of each model: how well the model could predict the exact next price, and how well the model could tell if the price would go up or down with respect to the last one. The RNN was better than the SVM and the ANN in both cases. Hence, the best performance was found in the RNN.

When comparing the ANN and the RNN between each other, we found that the mean square error of the RNN was lower and had less variance than the ANN's results. The MSE of the RNN after running the model 30 times was 0.61 lower than the ANN's MSE. The variance was also lower by 0.8 than the ANN's.

The SVM's performance was between the ANN's and the RNN's. It had a higher accuracy (44\%), just 1\% lower than the RNN, but a higher MSE (3.39) than the ANN (2.91) and the RNN (1.66). We also used the precision, recall, and F1 score to compare the models. 

The recall metric showed that the SVM was the best model to predict the downs of the price, while the RNN was the best to predict the ups. The F1 score reassured that the best performance was found in the RNN.

The results were congruent with the theory. The RNN respects the dependency between the inputs and, therefore, its performance should be better than the ANN's. The SVM can be seen as an optimization of the ANN model and, thus, should get better results than it. 

However, better performance was expected from the three models. Although the price predictions were very close from the real values, the second criteria's accuracy was considerably low. One reason may be the small number of examples at the training subset (167). In fact, the models had very little information to work with since the attributes for each one were just the three last prices. Stock market prices depend on a lot more variables.

For this specific application, the most serious errors are done when the model predicts a higher price and the real value goes down. In this case, the investor would lose money if he decided to sell the stock based on the model's prediction. From this perspective, the SVM is the less recommended option and the RNN is again the best one. 

\section{Future Work}

As stated in the scope of this work, the models are trained to predict the next price in the sequence. It would be interesting to have a model able to predict the next n prices rather than just the next one.

It would also be worth it to test the models with other company's stock market prices to compare the performance between each other. Another area of interest is to work with hybrid models. As we analyzed, each model is better than the others in different aspects. The idea is to train the three models and define a weighted average for the prediction depending on the performance of each model and the interests of the user.

Stock market price prediction is a difficult problem because all the random movement of the price and all the variables involved. The incorporation of news, reviews, and expectations should improve the performance of the model as well. 



	% \include{Chapters/relatedwork}

	% \include{Chapters/method}

	% \include{Chapters/results}
 
	% \include{Chapters/conclusions}

	%%%%%%%%%%%%%%%%%%%%%%%%%%%%%%%%%%%%%%%%%%%%%%
	% APPENDIX
	%%%%%%%%%%%%%%%%%%%%%%%%%%%%%%%%%%%%%%%%%%%%%%
	\appendix
	% \include{Chapters/appendixA}

	%%%%%%%%%%%%%%%%%%%%%%%%%%%%%%%%%%%%%%%%%%%%%%
	% BIBLIOGRAPHY
	%%%%%%%%%%%%%%%%%%%%%%%%%%%%%%%%%%%%%%%%%%%%%%
	\clearpage
	\addcontentsline{toc}{chapter}{References} %Añadimos la bibliografia a la lista de contenidos.
	
	%%%%%%%%% Referencias usando el sistema embedido %%%%%%%%%%%
	% e.g. (Ejemplo tomado de https://en.wikibooks.org/wiki/LaTeX/Bibliography_Management)
	%
	% \begin{thebibliography}{9}
	%
	%	\bibitem{lamport94}
    %			Leslie Lamport,
    %			\emph{LaTeX: a document preparation system},
    %			Addison Wesley, Massachusetts,
  	%			2nd edition,
    % 			1994.
    %
	% \end{thebibliography}

	%%%%%%%%% Referencias usando bibtex %%%%%%%%%%%
	\bibliographystyle{plain}
	\bibliography{references} 

\end{document}