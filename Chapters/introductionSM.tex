\chapter{Introduction}
\label{ch:introsm}

%\begin{chapterquote}{Oscar Wilde}
	%Where there is no love, there is no understanding.
%\end{chapterquote}


Forecasting is of great interest in different areas of scientific, industrial, and commercial activities. A time series is a collection of observations made sequentially through time \cite{chatfield2000time}. For example, the temperature in a specific region, the sales of a product in a store, and the electricity consumption in a fixed time interval. Forecasting stock prices has been regarded as one of the most challenging applications of modern time series forecasting \cite{pai2005hybrid}. It involves a lot of uncertainties and its nature is essentially dynamic, non-linear, complicated, nonparametric, and chaotic.

There are suggestions that it is not possible to predict stock prices because of its random behaviour. Nevertheless, technical analysts believe that most information about the stocks is reflected in recent prices. Therefore, if trends in the movements are observed then prices can be easily predicted \cite{patel2015predicting}. There are two types of analysis which usually investors perform before investing in a stock: fundamental analysis and technical analysis.

The fundamental analysis looks at intrinsic value of stocks, performance of the industry and economy, political climate etc.. On the other hand, technical analysis is the evaluation of stocks by means of studying statistics generated by market activity. For example, past prices and volumes. Technical analysts use stock charts to identify patterns and trends that may suggest how a stock will behave in the future\cite{patel2015predicting}. This work focuses on the second type of analysis.

Financial forecasting is of great interest for investors since it helps to decide to invest in a stock or not as well as to sell it or not in order to get profits and avoid losing money. A lot of research has been done around financial forecasting. There are two types of predictive models: traditional statistical models and machine learning models.

The first type includes the moving average, exponential smoothing, and the  autoregressive integrated moving average (ARIMA) models. These kind of models are linear in their predictions of the future values- They analyze historic data and attempt to approximate future values of a time series as a linear combination of these historic data \cite{shah2014performance}. Artificial Neural Networks (ANN) and Support Vector Regression (SVR) are two machine learning algorithms which have been most widely used for predicting stock price values  \cite{patel2015predicting}. Machine Learning uses a set of samples to generate an approximation of the underling function that generated the data \cite{shah2014performance}.

In \cite{shah2014performance} they compared the performance of Multilayer feed-forward neural network and the Elman neural network as the representative of the recurrent neural networks for stock prediction. They found that the neural network is better than the Elman recurrent network and suggested that the Elman model is only used for historical data and research purposes. On the other hand, in \cite{kim2003financial} a suport vector machine (SVM) model is applied to predicting the stock price index. Their experimental results showed that SVM provides a promising alternative to stock market prediction as they outperformed back-propagation neural networks (BPN).

Researchers have also tried hybrid models, meaning a combination of two or more models that can be machine learning and/or traditional statistical models. In \cite{pai2005hybrid} a hybrid model of autoregressive integrated moving average (ARIMA) and a support vector machine model is presented to solve the stock price forecasting problem. They found that a simple combination of the individual models does not necessarily produce the best results and demonstrated great interest in the structured selection of optimal parameters of the model.

Another hybrid model is presented in \cite{rather2015recurrent}. They proposed a model constituted of two linear models: ARIMA, and exponential smoothing model; and a non-linear model: recurrent neural network. Their results confirm about the accuracy of the prediction performance of recurrent neural network compared to linear models.

Finally, this work compares the performance of two machine learning models: artificial neural network , support vector machine, and a deep learning model: recurrent neural network (RNN) with long short term memory networks (LSTM). ANN and SVM have been widely used for stock market forecasting and RNN with LSTM is a good option for problems where data is non-linear as in this case, and where the patterns are difficult to be captured by traditional models.

\section{Objectives}
Design and implement a time series model for stock market price prediction using a recurrent neural network.

Compare the performance of the recurrent neural network model to two other predictive machine learning models, specifically, a feed forward artificial neural network model and a support vector machine model.

\section{Scope}
In this study a stock market price prediction model is presented. This approach does not seek to get a number of next predictions but just the next one. Meaning, it requires a number $n$ of real past predictions and determines the $n+1$ price in the sequence. 

Another limitation is that just machine learning models were taken in account. Traditional models such as ARIMA are not compared. Finally, stock market’s movements are affected by many macro-economical factors such as political events, general economic conditions, bank exchange rates, investors’ expectations, and so on. The data used for training this models just incorporates the prices of a fixed time interval.\\\\


The rest of this work is organized as follows. In the next chapter machine learning and deep learning are introduced. First, we explain the McCulloch and Pitts's Neuron model in order to easily understand the traditional neural networks. Then, we explain support vector machine as an optimization of neural networks.

Next, we move to deep learning and study the recurrent neural networks as well as some techniques to optimize them. This includes regularization, gradient descent variants and gradient descent optimization algorithms.  

In  chapter 3, we show how we can represent stock market prices as time series vectors and we also show the architecture for the three models. Then, we exhibit and analyze  the results of the experiments, and finally we draw the conclusions at the final chapter.

